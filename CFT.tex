\documentclass{article}

\usepackage{ctex}

\usepackage{graphicx}
\usepackage{amsmath}
\numberwithin{equation}{section}
\usepackage{amsfonts}
\usepackage{amssymb}
\allowdisplaybreaks[4]
\usepackage{braket}
\usepackage{slashed}

\usepackage{geometry}
\geometry{a4paper,left=2cm,right=2cm,top=2.5cm,bottom=2.5cm}

\usepackage{titlesec}
\titleformat
{\section}
{\centering\LARGE\bfseries\heiti}
{\arabic{section}.}
{0.5em}
{}[\titlerule]

\titleformat
{\subsection}[block]
{\Large\bfseries\heiti}
{\arabic{section}.\arabic{subsection}}
{0.5em}
{}[]

\titleformat
{\subsubsection}[block]
{\large\bfseries\heiti}
{\arabic{section}.\arabic{subsection}.\alph{subsubsection}}
{0.5em}
{}[]


\newcommand{\equ}[1]{(\ref{#1}) }


\title{\bfseries\Huge Conformal Field Theory}
\author{刘璇彰}
\date{}

\begin{document}
\maketitle
\section{Movitation}
对称性是研究宇宙自然规律的一个基本物理定律,通常来说一个物理系统的对称性越高,它的物理机制越简单。在场论中,对称性意味着在一个对称变换下物理系统的作用量不变,相应的运动方程也不变,{\heiti 可以证明,共形对称性是具有非平庸关联函数的理论所能拥有的最大对称性。}而研究具有共形对称性的经典或量子场论,就是所谓的{\heiti 共形场论(Conformal Field Theory)。}
\section{Path Integral}
量子场论初级课程中我们学习了利用对易关系来量子化场论的正则量子化,这一过程颇为不易,但我们真正关心的,或者说在计算散射截面中真正用到是后续的内容,如两点关联函数、费曼传播子、费曼规则等。
\par
与正则量子化对应的路径积分表述(Path Integral Formalism)是一种更加“物理”的量子化方案,它最初的建立是为了让量子力学和经典力学联系更加紧密,但创立之后迅速发展为量子场论最基本最常用的形式。本章将从最简单的一维量子力学出发,导出场论的路径积分形式,并简单讨论标量场和Dirac场的路径积分量子化细节。
\subsection{一维量子力学}
考虑一个具有哈密顿量为$\hat{H}$的粒子,它从初态$\ket{q,t}$演化至末态$\ket{q',t'}$的跃迁矩阵元表示为
\begin{align}
	\Braket{q',t'|q,t} = \Braket{q',t'|e^{-i\hat{H} \left(t'-t\right)}|q,t}\label{Basic}
\end{align}
但仔细观察这个矩阵元时会发现,由于指数中含坐标的势能部分和含动量的动能部分不对易,造成计算上的巨大困难,为了解决这一困境,我们将$(t,t')$的时间间隔分为 $n$ 个等份,记每一份为
\begin{align}
	\delta t = \frac{t'-t}{n}\label{deltat}
\end{align}
也就将\equ{Basic}改写为
\begin{align*}
	\Braket{q',t'|q,t} = \Braket{q',t'|e^{-i n \hat{H} \delta t}|q,t}
\end{align*}
我们在每一个$\delta t$后插入一组态矢量的完备基,得到
\begin{align}
	\Braket{q',t'|q,t} = \int \mathrm{d}q_1 \ldots \mathrm{d}q_{n-1}&\Braket{x'|e^{-i\hat{H} \delta t}|q_{n-1}}\Braket{q_{n-1}|e^{-i\hat{H} \delta t}|q_{n-2}} \notag \\
	\ldots &\Braket{q_1|e^{-i\hat{H} \delta t}|q}
\end{align}
如果每个$\delta t$足够小,也就是 $\lim n \rightarrow \infty$时,每个矩阵元可以近似写为
\begin{align}
	\Braket{q_j|e^{-i\hat{H} \delta t}|q_i}=\Braket{q_j|\left(1-iH \delta t \right)|q_i}+O\left({\delta t }^2\right)
\end{align}
如果哈密顿量具有经典形式$\hat{H}=\frac{\hat{p}^2}{2m}+V(\hat{x})$,结合$ \Braket{q_j|q_i}=\delta \left( q_j - q_i\right)$和$\Braket{q|p}=e^{ipq}$,可以得到
\begin{align*}
	\Braket{q_j|\hat{H}|q_i}&=\Braket{q_i|\frac{\hat{p}}{2m}|q_i}+V \left(\frac{q_i+q_j}{2} \right)\delta \left( q_i - q_j\right)\\
	&=\int \frac{\mathrm{d}p}{2\pi} \Braket{q_j|p} \Braket{p|\frac{\hat{p}}{2m}|q_i}+V \left(\frac{q_i+q_j}{2} \right)\int \frac{\mathrm{d}p}{2\pi} e^{ip \left( q_j - q_i\right)}\\
	&=\int \frac{\mathrm{d}p}{2\pi} e^{ip \left( q_j - q_i\right)}\left[ \frac{p^2}{2m}+ V \left(\frac{q_i+q_j}{2} \right)\right]
\end{align*}
那么我们可以推出
\begin{align}
	\Braket{q_j|e^{-i\hat{H} \delta t}|q_i} & \approx \int \frac{\mathrm{d}p}{2\pi} e^{ip \left( q_j - q_i\right)} \left\{1 - i\delta t \left[ \frac{p^2}{2m}+ V \left(\frac{q_i+q_j}{2} \right)\right]\right\} \notag \\
	&\approx \int \frac{\mathrm{d}p}{2\pi} e^{ip \left( q_j - q_i\right)} e^{-i\delta t H \left( p,\frac{q_j+q_i}{2}\right)}
\end{align}
于是,总的跃迁矩阵元可以写为
\begin{align}
	\Braket{q'|e^{-i\hat{H} \left( t'-t\right)}|q} \equiv & \int \left( \frac{\mathrm{d}p_1}{2\pi}\right) \ldots \left( \frac{\mathrm{d}p_n}{2\pi}\right)\int \mathrm{d}q_1 \ldots \mathrm{d} q_{n-1} \notag \\
	& exp \left\{ i \left[\sum_{i=1}^n p_i \frac{\left( q_i - q_{i-1}\right)}{\delta t}\delta t- H \left( p_i,\frac{q_i+q_{i-1}}{2}\right)\delta t\right]\right\}
\end{align}
当$\lim n\rightarrow \infty$时,\equ{basic integral}可以写为
\begin{align}
	\Braket{q'|e^{-i\hat{H} \left( t'-t\right)}|q}  =& \lim_{n\rightarrow\infty} \int \left( \frac{\mathrm{d}p_1}{2\pi}\right) \ldots \left( \frac{\mathrm{d}p_n}{2\pi}\right)\int \mathrm{d}q_1 \ldots \mathrm{d} q_{n-1} \notag \\
	& exp \left\{ i \left[\sum_{i=1}^n p_i \frac{\left( q_i - q_{i-1}\right)}{\delta t}\delta t- H \left( p_i,\frac{q_i+q_{i-1}}{2}\right)\delta t\right]\right\}\label{basic integral} \\
	=& \int \left[ \frac{\mathrm{d}p\mathrm{d}q}{2\pi} \right] exp \left\{ i\int_{t}^{t'} \mathrm{d} t \left[ p\dot{q} -H \left( p,q\right) \right] \right\}\label{define}
\end{align}
如果再考虑到通常情况下$H$都是$p$ 的二次型,那\equ{define}就是关于$p$的高斯积分,利用
\begin{align*}
\int_{-\infty}^{+\infty}\frac{\mathrm{d}x}{2\pi}e^{-ax^2+bx}=\frac{1}{\sqrt{4\pi a}}e^{\frac{b^2}{4a}}
\end{align*}
可以将\equ{basic integral}改写为
\begin{align}
	\Braket{q'|e^{-i\hat{H} \left( t'-t\right)}|q}  =& \lim_{n\rightarrow\infty} {\left( \frac{m}{2\pi i \delta t}\right)}^{\frac{n}{2}}\int \prod_{i=1}^{n-1}\mathrm{d}q_i exp \left\{i \sum_{i=1}^{n}\delta t \left[ \frac{m}{2}{\left( \frac{q_i-q_{i-1}}{\delta t}\right)}^2 -V\right]\right\}\\
	=& N \int \mathcal{D}q\ exp\left\{i\int_{t}^{t'}\mathrm{d}\tau \left[ \frac{m\dot{q}^2}{2}-V \left( q\right) \right]\right\}\notag \\
	=& N\int \mathcal{D}q\ e^{iS}\label{formalism}
\end{align}
\equ{formalism}是路径积分中最常用的公式,它的推导过程表明,要想避开复杂的算符运算,就必须付出无穷多积分的代价,但需要明确的是,算符运算和路径积分并没有严格的优劣之分,前者在算符展开中更易操作,后者更适合应对有约束的系统,如规范场论。
上述的论证是对普通的量子力学进行的,对于量子场论可以用完全相似的方式得出它们的路径积分形式,最后的表达式只需要简单地做一些替换即可
\begin{align*}
	\mathcal{D}q\mathcal{D}p&\Rightarrow \mathcal{D}\phi \left( x\right) \mathcal{D}\pi \left( x\right)\\
	L \left( q,\dot{q}\right)&\Rightarrow \int \mathcal{L}\left( \phi,\partial_{\mu}\phi \right)\mathrm{d}x^3\\
	H \left( p,q\right)&\Rightarrow \int \mathcal{H}\left( \phi,\pi \right)\mathrm{d}x^3
\end{align*}
\subsection{标量场的路径积分量子化}
在这一节我们将利用路径积分的形式量子化实标量场,并最终导出标量场的费曼规则。
我们已经熟知标量场的哈密顿量为
\begin{align*}
     H = \int \mathrm{d}^3x \left[ \frac{1}{2}\pi^2 +\frac{1}{2}\left( \nabla\phi\right)^2 + V \left( \phi\right) \right]
\end{align*}
则对于一个标量场从$\phi_a \left( \mathbf{x}\right),x^0_a=-T$ 演化到$\phi_b \left( \mathbf{x}\right),x^0_b=T$,我们可以直接写出它的路径积分量子化形式
\begin{align}
	\Braket{\phi_b \left( \mathbf{x}\right)|e^{-iHT}|\phi_a \left( \mathbf{x}\right)}=\int \mathcal{D}\phi \ exp \left[ i \int_{-T}^{T}\mathrm{d}^4 \mathcal{L} \right]
\end{align}
接着,我们来计算一下相互作用场论中的两点关联函数,考虑这样一个在时间$T$ 和 $-T$处有给定的边界条件 $\phi \left( -T\right)=\phi_a$ 和$\phi \left( T\right)=\phi_b$的式子
\begin{align}
	\int \mathcal{D}\phi \left( x\right)\ \phi\left( x_1\right) \phi\left( x_2\right)exp \left[ i\int_{-T}^{T} \mathrm{d}^4 x \mathcal{L}\right]\label{scalar correlation}
\end{align}
下面,我们下面来证明\equ{scalar correlation}就是我们熟知的两点关联函数$ \Braket{\Omega|T \left[ \phi_H \left( x_1\right) \phi_H \left( x_2\right)\right]|\Omega}$。首先,将完整的$\int \mathcal{D}\phi$拆分成 $\int\mathcal{D}\phi_1 \int\mathcal{D}\phi_2 \int_{\phi \left( x_{1,2}^0,\mathbf{x}\right)=\phi_{1,2} \left( \mathbf{x}\right)}\mathcal{D}\phi \left( \mathbf{x}\right)$,这个拆分意味着在原有的每一时刻所有路径的积分中,固定时刻$x_1^0$和 $ x_2^0$处的取值(对它们单独积分)并保留其余各处的积分。不失一般性,假设$ x_2^0 > x_1^0$,那么原式就可以写为
\begin{align}
	&\int\mathcal{D}\phi_1\ \phi_1 \left( x_1\right) \int\mathcal{D}\phi_2 \ \phi_2 \left( x_2\right)\int_{\phi \left( x_{1,2}^0,\mathbf{x}\right)=\phi_{1,2} \left( \mathbf{x}\right)}\mathcal{D}\phi \left( \mathbf{x}\right)exp \left[ i \int_{-T}^T \mathrm{d}^4 x \mathcal{L} \right]\notag \\
	=&\int\mathcal{D}\phi_1\ \phi_1 \left( x_1\right) \int\mathcal{D}\phi_2 \ \phi_2 \left( x_2\right) \Braket{\phi_b|e^{-iH \left( T-x_2^0\right)}|\phi_2}\Braket{\phi_2|e^{-iH \left( x_2^0-x_1^0\right)}|\phi_1}\Braket{\phi_1|e^{-iH \left(x_1^0+T\right)}|\phi_2}\notag \\
	=&\Braket{\phi_b|e^{-iH \left( T-x_2^0\right)}\phi_s \left( x_2\right)e^{-iH \left( x_2^0-x_1^0\right)}\phi_s \left( x_1\right)e^{-iH \left( x_1^0+T\right)}|\phi_a}\notag \\
	=& \Braket{\Omega|e^{-iHT}\phi_H \left( x_1\right)\phi_H \left( x_2\right)e^{-iHT}|\Omega}\label{correlation1}
\end{align}
如果在\equ{correlation1}中调转$x_2^0$和 $ x_1^0$的顺序,通过同样的步骤也能的到正确的关联函数,这意味着路径积分的方法是自动编时的。最后利用计算相互作用场论真空态$ \ket{\Omega}$时使用的技巧,令时间$T$在趋于无穷时有一个很小的虚部$T\rightarrow \infty \left( 1-i\epsilon\right)$,以消除非基态能的影响,即$e^{-iHT}=\sum\limits_{n} e^{-iE_nT}\ket{n} \Braket{n|\phi_a}\xrightarrow[T\rightarrow \infty \left( 1-i\epsilon\right)]{} \Braket{\Omega|\phi_a}e^{-iE_o \cdot \infty \left( 1-i\epsilon\right)}\ket{\Omega}$,由此可得到两点关联函数表达式
\begin{align}
	\Braket{\Omega|T \left[ \phi_H \left( x_1\right) \phi_H \left( x_2\right)\right]|\Omega}=\lim_{T \rightarrow \infty \left( 1-i\epsilon\right)}\frac{\int \mathcal{D}\phi \ \phi \left( x_1\right)\phi \left( x_2\right)exp \left[ i\int_{-T}^{T}\mathrm{d}^4\mathcal{L} \right]}{\int \mathcal{D}\phi \ exp \left[ i\int_{-T}^{T}\mathrm{d}^4\mathcal{L} \right]}\label{correlation2}
\end{align}
这一公式可以自然地推广至多点关联函数。
\subsection{生成泛函与标量场费曼规则}
为了用路径积分表述得到费曼规则,我们可以暴力计算\equ{correlation2},先通过离散化得到表达式再取连续极限,但这一过程极度繁琐,Peskin甚至直言这一做法是“ugly”的,所以我们绕过这一蛮力的方法,直接引入更普适简洁的生成泛函。

\subsubsection{生成泛函}
首先介绍泛函导数(Functional Derivative)的定义
\begin{align*}
	\frac{\delta F \left[ J \left( x\right) \right]}{\delta J \left( y\right)} =\lim_{\epsilon\rightarrow 0}\frac{F \left[ J \left( x\right)+\epsilon \delta \left( x-y\right) \right]-F \left[ J \left( x\right) \right]}{\epsilon}
\end{align*}
由此可以得到两个重要的泛函求导公式
\begin{align}
	\begin{split}
		\begin{cases}
			&\frac{\delta}{\delta J \left( x\right)}J \left( y\right)=\delta^{\left( 4\right)}\left( x-y\right)\\
			&\frac{\delta}{\delta J \left( x\right)}\int \mathrm{d}^4yJ \left( y\right)\phi \left( y\right)=\phi \left( x\right)
		\end{cases}
	\end{split}
\end{align}
所以我们可以得到
\begin{align}
	\frac{\delta}{\delta J \left( x\right)}\ exp \left[ i \int \mathrm{d}^4 J \left( x\right) \phi \left( x\right) \right]=i\phi \left( x\right) exp \left[ i \int \mathrm{d}^4 J \left( x\right) \phi \left( x\right) \right]\label{gener1}
\end{align}
根据\equ{gener1},我们可以定义标量场的生成泛函(Generating Functional)$Z \left[ J \right]$
\begin{align}
	Z \left[ J \right] \equiv \mathcal{D}\phi \ exp \left\{ i \int \mathrm{d}^4 y \left[ \mathcal{L} + J \left( y\right)\phi \left( y\right) \right] \right\}
\end{align}
当$J=0$ 时,定义
\begin{align}
	Z_0=Z \left[ 0 \right]=\int \mathcal{D}\phi \ exp \left[ i\int \mathrm{d}^4 y \mathcal{L} \right]
\end{align}
可以验证,定义生成泛函之后可以将两点关联函数(自由场论)改写为较为简洁的形式
\begin{align}
	\Braket{0|T \left[ \phi_H \left( x_1\right) \phi_H \left( x_2\right)\right]|0}&=\frac{\int \mathcal{D}\phi \ \phi \left( x_1\right)\phi \left( x_2\right)exp \left[ i\int_{-T}^{T}\mathrm{d}^4\mathcal{L} \right]}{\int \mathcal{D}\phi \ exp \left[ i\int_{-T}^{T}\mathrm{d}^4\mathcal{L} \right]}\notag \\
	&\left.=\frac{1}{Z_0}\left( -i \frac{\delta}{\delta J \left( x_1\right)}\right)\left( -i \frac{\delta}{\delta J \left( x_2\right)}\right)Z \left[ J \right]  \right|_{J=0}\label{derivative}
\end{align}
以实标量场为例,其生成泛函为
\begin{align}
	Z \left[ J \right] =\int \mathcal{D}\phi\  exp \left[ i\int \mathrm{d}^4 x \left[ \mathcal{L} J \left( x\right)\phi \left( x\right)\right] \right]\label{function0}
\end{align}
对指数部分进行一次分部积分,有
\begin{align}
	\int \mathrm{d}^4x \left( \mathcal{L}_0 \left( \phi\right) +J \left( \phi\right) \right) =\int \mathrm{d}^4x \left( \frac{1}{2} \phi\left( -\partial^2-m^2+i\epsilon \right)\phi+J\phi\right)\label{function1}
\end{align}
对$\phi$添加一项常数项
\begin{align*}
     \phi' \left( x\right) = \phi \left( x \right) - i\int \mathrm{d}^4 y D_F \left( x-y \right)J \left( y\right)
\end{align*}
带入\equ{function1}得到
\begin{align}
	\int \mathrm{d}^4x \left( \mathcal{L}_0 \left( \phi\right) +J \left( \phi\right) \right) &=\int \mathrm{d}^4x \left[ \frac{1}{2} \phi'\left( -\partial^2-m^2+i\epsilon \right)\phi'+J \left( -\partial^2-m^2+i\epsilon\right)^{-1}J\right]\notag \\
	&=\int \mathrm{d}^4x \left[ \frac{1}{2} \phi'\left( -\partial^2-m^2+i\epsilon \right)\phi'+J \left( -iD_F\right)J\right]		\label{function2}
\end{align}
由于只是给$\phi$添加一项常数,积分的雅克比行列式为$1$,将\equ{function2}带回\equ{function0}得到
\begin{align}
	\int \mathcal{D} \phi' \ exp \left[ i\int \mathrm{d}^4 x\mathcal{L}_0 \left( \phi'\right) \right]exp \left\{ -i\int \mathrm{d}^4 x \mathrm{d}^4y \frac{1}{2} J \left( x\right) \left[ -i D_f \left( x-y\right) \right]J \left( y\right)\right\}\label{function3}
\end{align}
由于\equ{function3}中第二个指数内的函数与$\phi'$ 无关,所以可将第一个指数对$\phi'$积分得到 $Z_0$,则最后化简为
\begin{align}
	Z \left[ J \right] = Z_0 exp \left[ -\frac{1}{2}\int \mathrm{d}^4x \mathrm{d}^4 y J \left( x\right) D_F \left( x-y\right)J \left( y\right) \right]\label{function4}
\end{align}
最后利用\equ{derivative}和\equ{function4},得到
\begin{align}
	&\Braket{0|T \left[ \phi \left( x_1\right) \phi \left( x_2\right)\right]|0}\notag \\
	=& - \frac{\delta}{\delta J \left( x_1\right)}\frac{\delta}{\delta J \left( x_2\right)} exp\left. \left[ -\frac{1}{2} \int \mathrm{d}^4x \mathrm{d}^4 y J \left( x\right) D_F \left( x-y\right)J \left( y\right)\right]\right|_{J=0}\notag \\
	=&- \left. \frac{\delta}{\delta J \left( x_1\right)}\left[ -\frac{1}{2}\int \mathrm{d}^4y D_F \left( x_2-y\right)J \left( y\right) -\frac{1}{2}\int \mathrm{d}^4 x J \left( x\right) D_F \left( x-x_2\right) \right]\frac{Z \left[ J \right]}{Z_0}\right|_{J=0}\notag \\
	=& D_F \left( x_1-x_2\right)
\end{align}

\subsubsection{费曼规则}
利用同样的思路,对于四点关联函数有
\begin{align}
	&\Braket{0|T \left[ \phi \left( x_1\right) \phi \left( x_2\right)\phi \left( x_3\right)\phi \left( x_4\right)\right]|0}\notag \\
	=& \left.\frac{\delta}{\delta J \left( x_1\right)} \frac{\delta}{\delta J \left( x_2\right)} \frac{\delta}{\delta J \left( x_3\right)} \frac{\delta}{\delta J \left( x_4\right)}exp \left[ -\frac{1}{2}J \left( x\right)D_F \left( x-y\right)J \left( y\right) \right]\right|_{J=0}\notag \\
	=&\left.\frac{\delta}{\delta J \left( x_1\right)} \frac{\delta}{\delta J \left( x_2\right)} \frac{\delta}{\delta J \left( x_3\right)}\left[ -J \left( x\right)D_F \left( x-x_4\right) \right]exp \left[ -\frac{1}{2}J \left( x\right)D_F \left( x-y\right)J \left( y\right) \right]\right|_{J=0}\notag \\
	=&\left.\frac{\delta}{\delta J \left( x_1\right)} \frac{\delta}{\delta J \left( x_2\right)} \left[-D_F \left( x_3-x_4\right) +J \left( x\right)D_F \left( x-x_4\right)J \left( y\right) D_F \left( x-y_3\right) \right]exp \left[ -\frac{1}{2}J \left( x\right)D_F \left( x-y\right)J \left( y\right) \right]\right|_{J=0}\notag \\
	=&\frac{\delta}{\delta J \left( x_1\right)}\left[D_F \left( x_3-x_4\right)J \left( x\right)D_F \left( x-x_2\right)+D_F \left( x_2-x_4\right)J \left( y\right)D_F \left( x-y_3\right)+J \left( x\right)D_F \left( x-x_4\right)D_F \left( x_2-x_3\right)\right]\notag\\
	&\left.exp \left[ -\frac{1}{2}J \left( x\right)D_F \left( x-y\right)J \left( y\right) \right]\right|_{J=0}\notag \notag\\
	=& D_F \left( x_3-x_4\right)D_F \left( x_1-x_2\right)+D_F \left( x_2-x_4\right)D_F \left( x_1-x_3\right)+D_F \left( x_1-x_4\right)D_F \left( x_2-x_3\right)
\end{align}
符合我们的预期
\subsection{Dirac场的路径积分量子化}
我们已经知道狄拉克场遵循反对易关系,为了的到它的路径积分形式,我们首先详细介绍一下满足反对易关系的通用代数性质,也就是Grassmann numbers
\subsubsection{Grassmann 代数}
Grassmann数的基本性质是任意两个数$\theta,\eta$之间满足反对易关系
\begin{align}
    \theta \eta +\eta \theta =0
\end{align}
由于$\theta,\eta$的选取完全任意,显然有$\theta^2=0$,因此更高次幂的项也自然为零,于是任意一个定义在Grassmann数上的函数只能写为
\begin{align}
    f \left( \theta \right)=A + B\theta
\end{align}
其中$A,B$为一般的对易数。在此基础上,我们考虑Grassmann数上的积分,由于它肯定无法适用一般的积分,我们考虑“积分”操作的一般特点:
\begin{align*}
	&\text{1.线性:}\int \mathrm{d}\theta\ \left[af \left( \theta\right) +bg \left( \theta\right) \right]=a\int \mathrm{d}\theta f \left( \theta\right)+b\int \mathrm{d}\theta g \left( \theta\right)\\
	&\text{2.在平移操作下积分不变:}\int \mathrm{d} \left( A+B\theta\right) = \int \mathrm{d}\theta\left( \left( A+B\eta\right)+B\theta\right)
\end{align*}
综合这两个特点,可以总结出,对于Grassmann数的积分只能是$B$ 的倍数$KB$,一般而言,可以将这个系数设置为$1$,利用这样的结果,我们可以得到
\begin{align}
     \begin{cases}
          \begin{split}
				&\int \mathrm{d}\theta =1\\
				&\int \mathrm{d}\theta\ \theta =1
          \end{split}
     \end{cases}
\end{align}
由于Grassmann数的反对易性,多重积分的顺序是很重要的,例如
\begin{align}
    \int \mathrm{d}\theta \int \mathrm{d} \eta \mathrm{d}\theta \ \eta\theta = 1
\end{align}
接着来考虑Grassmann的厄密共轭
\begin{align}
     \left( \theta \eta\right)^*=\eta^* \theta^*=-\theta^* \eta^*
\end{align}
可以定义
\begin{align}
	\theta=\frac{\theta_1+i\theta_2}{\sqrt{2}}\ \ ,\ \ \theta^*=\frac{\theta_1-i\theta_2}{\sqrt{2}}
\end{align}
且有
\begin{align}
     \int \mathrm{d}\theta^*\mathrm{d}\theta \left( \theta \theta^*\right)=1
\end{align}
而对于高斯积分
\begin{align}
     \begin{cases}
         \begin{split}
			&\int \mathrm{d}\theta^*\mathrm{d}\theta\  e^{-\theta^*b\theta} =\int \mathrm{d}\theta^*\mathrm{d}\theta \left( 1-\theta^*b\theta\right)=\int \mathrm{d}\theta^*\mathrm{d}\theta \left( 1+\theta\theta^*b\right)=b\\
			&\int \mathrm{d}\theta^*\mathrm{d}\theta\ \theta\theta^* e^{-\theta^*b\theta} =\int \mathrm{d}\theta^*\mathrm{d}\theta\ \theta\theta^* \left( 1-\theta^*b\theta\right)=\int \mathrm{d}\theta^*\theta\ \theta\theta^*=1
         \end{split}
     \end{cases}
\end{align}
\subsubsection{Dirac场的量子化}
有了Grassmann数,我们可以对旋量场进行量子化。考虑一个费米子场$\Psi \left( x\right)$,可以将它表示为
\begin{align}
     \Psi \left( x\right) = \sum_i \Psi_i \phi_i \left( x\right)
\end{align}
其中$\phi_i \left( x\right)$为普通的对易数值构成的场,而为了描述Dirac场,取它为四分量旋量的基底,$\Psi_i$为Grassmann系数。基于此,我们可以类比标量场写出Dirac场的两点关联函数
\begin{align}
	\Braket{0|T \left[ \Psi \left( x_1\right)\bar{\Psi} \left( x_2\right) \right]|0} =\frac{\int \mathcal{D}\bar{\Psi}\mathcal{D}\Psi\ exp \left[ i \int \mathrm{d}^4 x \bar{\Psi}\left( i\slashed{\partial}-m\right)\Psi \right]\Psi \left( x_1\right)\Psi \left( x_2\right)}{\int \mathcal{D}\bar{\Psi}\mathcal{D}\Psi\  exp \left[ i \int \mathrm{d}^4 x \bar{\Psi}\left( i\slashed{\partial}-m\right)\Psi \right]}
\end{align}
定义Dirac场的生成泛函为
\begin{align}
	Z \left[ \bar{\eta},\eta \right]=\int \mathcal{D}\bar{\Psi}\mathcal{D}\Psi\ exp \left[ i\int \mathrm{d}^4 \left[ \bar{\Psi} \left( i\slashed{\partial}-m\right)\Psi +\bar{\eta}\Psi +\bar{\Psi}\eta \right] \right]\label{Dirac two point}
\end{align}
同样类比标量场,对$\Psi$和 $\bar{\Psi}$做一次shift,各自添加一个常数项
 \begin{align}
    \begin{cases}
        \begin{split}
			\Psi = \Psi' -i \int \mathrm{d}^4y S_F \left( x-y\right)\eta \left( y\right) \\
			\bar{\Psi} = \bar{\Psi'} +i\int\mathrm{d}^4y S_F^* \left( x-y\right)\eta \left( y\right)\label{Dirac shift}
        \end{split}
    \end{cases}
\end{align}
将\equ{Dirac shift}代入\equ{Dirac two point}得到
\begin{align}
	Z \left[ \bar{\eta},\eta \right]&=\int \mathcal{D}\bar{\Psi}\mathcal{D}\Psi\ exp \left[ i\int \mathrm{d}^4 \left[ \bar{\Psi} \left( i\slashed{\partial}-m\right)\Psi +\bar{\eta}\Psi +\bar{\Psi}\eta \right] \right]\notag \\
	&=\int \mathcal{D}\Psi'\mathcal{D}\bar{\Psi'}\ exp \left[ i \int \mathrm{d}x \left[ \bar{\Psi'}\left( i\slashed{\partial}-m\right) \Psi'\right] \right]exp \left[ -\int \mathrm{d}^4 x \mathrm{d}^4 y \eta \left( x\right)S_F \left( x-y\right)\eta \left( y\right) \right]\notag \\
	&= Z_0 exp \left[ -\int \mathrm{d}^4x \mathrm{d}^4y \bar{\eta}\left( x\right)S_F \left( x-y\right)\eta \left( y\right) \right]
\end{align}
其中$\eta$是一个Grassmann场。最后利用Grassmann数的求导法则
\begin{align*}
    \frac{\mathrm{d}}{\mathrm{d} \eta}\theta \eta=-\frac{\mathrm{d}}{\mathrm{d} \eta}\eta\theta=-\theta
\end{align*}
就可以把Dirac场的两点关联函数写为泛函导数的形式
\begin{align}
	\left.\Braket{0|T \left[ \Psi \left( x_1\right)\Psi \left( x_2\right) \right]|0} =\frac{1}{Z_0}\left( -i \frac{\delta}{\delta \bar{\eta }\left( x_1\right)}\right)\left( -i \frac{\delta}{\delta \eta \left( x_2\right)}\right)Z \left[ \bar{\eta},\eta \right]\right|_{\bar{\eta},\eta=0}
\end{align}
\section{Conformal Group}
在这一章,我们将详细讨论所谓“共形对称性”的具体含义,类似于我们讨论量子场论的洛伦兹不变性,我们将首先找出共形变换的一般形式,研究无限小共形变换的性质,给出共形变换群的李代数,最后主要探讨一下二维场论中的共形变换理论。
\subsection{共形变换}
在复变函数中我们学过,对于一个解析函数$f \left( z\right)$,在它一阶导数不为$0$的时候 是一个“保角变换”,也就是定义域中两条线相交的夹角在值域中保持不变,而研究共形变换也就是研究“保角变换”的一般形式。
\subsection{$n \geq 3$ Conformal Group }


\end{document}